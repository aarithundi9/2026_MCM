\documentclass[12pt,letterpaper]{article}
\usepackage[margin=1in]{geometry}
\usepackage{amsmath,amssymb,amsthm}
\usepackage{graphicx}
\usepackage{booktabs}
\usepackage{array}
\usepackage{hyperref}
\usepackage{xcolor}
\usepackage{algorithm}
\usepackage{algpseudocode}
\usepackage{enumitem}
\usepackage{float}
\usepackage{fancyhdr}
\usepackage{titlesec}
\usepackage{multirow}

% Custom colors
\definecolor{dwtsblue}{RGB}{70,130,180}
\definecolor{dwtscoral}{RGB}{255,127,80}
\definecolor{warnorange}{RGB}{255,165,0}
\definecolor{successgreen}{RGB}{46,139,87}

% Page style
\pagestyle{fancy}
\fancyhf{}
\rhead{MCM Problem C 2026}
\lhead{Objective 3: Pro Dancer \& Celebrity Characteristics}
\rfoot{Page \thepage}

% Title formatting
\titleformat{\section}{\Large\bfseries\color{dwtsblue}}{\thesection}{1em}{}
\titleformat{\subsection}{\large\bfseries\color{gray}}{\thesubsection}{1em}{}

\title{\textbf{Dancing with the Stars: Pro Dancer \& Celebrity Characteristics} \\ 
\large MCM Problem C 2026 -- Objective 3 Technical Documentation}
\author{Team Documentation}
\date{January 2026}

\begin{document}

\maketitle

\tableofcontents
\newpage

%==============================================================================
\section{Executive Summary}
%==============================================================================

This document presents our analysis framework for quantifying how professional dancer and celebrity characteristics influence competition outcomes on \textit{Dancing with the Stars}. We developed novel methodologies including the \textbf{Pro Dancer Impact Score (PDIS)} and \textbf{Celebrity Archetype Clustering} to provide actionable insights for optimal pairing decisions.

\subsection*{Key Findings}

\begin{itemize}
    \item \textbf{Pro dancer variance contribution:} ICC = 5.3\% (pro dancer explains 5.3\% of placement variance)
    \item \textbf{Causal impact:} Top-tier pros improve placement by \textbf{1.70 places} on average (ATE via propensity matching)
    \item \textbf{Celebrity-only predictability:} $R^2 = 26.2\%$ -- celebrity characteristics alone explain about a quarter of performance
    \item \textbf{Three distinct celebrity archetypes:} Fast Learners, Underdogs, and Veteran Stars
    \item \textbf{Top PDIS Pros:} Daniella Karagach (100.0), Lindsay Arnold (88.0), Derek Hough (82.3)
\end{itemize}

\colorbox{successgreen!20}{\parbox{\dimexpr\linewidth-2\fboxsep}{
\textbf{Headline Result:} The professional dancer matters, but celebrity characteristics dominate. Our PDIS metric identifies which pros consistently outperform expectations, providing a data-driven pairing recommendation system.
}}

%==============================================================================
\section{Problem Formulation}
%==============================================================================

\subsection{Research Questions}

Objective 3 addresses three interconnected questions:

\begin{enumerate}
    \item \textbf{Variance Decomposition:} How much of placement variance is attributable to the professional dancer vs. celebrity characteristics?
    \item \textbf{Causal Inference:} Controlling for celebrity characteristics, what is the causal impact of being paired with a top professional?
    \item \textbf{Pairing Optimization:} Can we identify optimal pro-celebrity matches based on archetype compatibility?
\end{enumerate}

\subsection{Methodological Challenges}

\subsubsection{Selection Bias}
Producers strategically assign celebrities to professionals, creating endogeneity:
\begin{itemize}
    \item High-profile celebrities often paired with experienced pros
    \item Athletic celebrities may receive pros specializing in technical styles
    \item This confounding makes naive comparisons misleading
\end{itemize}

\subsubsection{Limited Sample Sizes}
With 60 unique professional dancers across 34 seasons, some pros have very few partnerships, making individual effect estimation noisy.

\subsection{Our Approach}

We employ a multi-pronged strategy:
\begin{enumerate}
    \item \textbf{Mixed-Effects Models:} Partition variance while controlling for celebrity features
    \item \textbf{Propensity Score Matching:} Isolate causal effect by matching similar celebrities
    \item \textbf{Counterfactual Analysis:} Develop PDIS by comparing actual vs. expected performance
    \item \textbf{Clustering:} Identify celebrity archetypes for pairing recommendations
\end{enumerate}

%==============================================================================
\section{Feature Engineering}
%==============================================================================

\subsection{Star Power Index (SPI)}

We developed a novel \textbf{Star Power Index} using Wikipedia page view data as a proxy for celebrity fame. For each celebrity $i$:

\begin{equation}
\text{SPI}_i = \log_{10}\left(1 + \frac{\text{PageViews}_{30\text{day}}}{\text{MedianPageViews}}\right)
\end{equation}

where page views are retrieved from the Wikimedia API for the 30-day period preceding their season's airdate.

\subsubsection{Rationale}
\begin{itemize}
    \item Wikipedia page views correlate with public awareness
    \item Available for historical analysis (unlike social media metrics)
    \item Log-transformation handles extreme values (e.g., A-list celebrities)
\end{itemize}

\subsubsection{SPI Distribution}
\begin{center}
\begin{tabular}{lcc}
\toprule
\textbf{Star Power Level} & \textbf{SPI Range} & \textbf{Count} \\
\midrule
Low Fame & 0--5 & 142 \\
Medium Fame & 5--10 & 189 \\
High Fame & 10--20 & 73 \\
Superstar & $>$20 & 17 \\
\bottomrule
\end{tabular}
\end{center}

\subsection{Celebrity Feature Set}

We engineered the following features for each celebrity:

\begin{center}
\begin{tabular}{lll}
\toprule
\textbf{Feature} & \textbf{Type} & \textbf{Description} \\
\midrule
\texttt{celebrity\_age} & Numeric & Age during season \\
\texttt{industry\_cluster} & Categorical & Entertainment, Sports, Media, Other \\
\texttt{region} & Categorical & North America, Europe, etc. \\
\texttt{star\_power\_index} & Numeric & Wikipedia-based fame metric \\
\texttt{first\_week\_score} & Numeric & Initial judge score (proxy for baseline skill) \\
\texttt{improvement\_rate} & Numeric & Week-over-week score growth \\
\texttt{score\_volatility} & Numeric & Standard deviation of weekly scores \\
\bottomrule
\end{tabular}
\end{center}

\subsection{Pro Dancer Feature Set}

For each professional dancer:

\begin{center}
\begin{tabular}{lll}
\toprule
\textbf{Feature} & \textbf{Type} & \textbf{Description} \\
\midrule
\texttt{pro\_experience} & Numeric & Number of seasons on DWTS \\
\texttt{win\_rate} & Numeric & Proportion of wins (Mirrorball trophies) \\
\texttt{finals\_rate} & Numeric & Proportion reaching finale \\
\texttt{avg\_placement} & Numeric & Mean placement across partnerships \\
\texttt{partnership\_count} & Numeric & Total celebrity partnerships \\
\bottomrule
\end{tabular}
\end{center}

\subsection{Performance Trajectory Features}

To capture learning dynamics, we computed:

\begin{equation}
\text{ImprovementRate}_i = \frac{\sum_{w=2}^{W} (\text{Score}_{i,w} - \text{Score}_{i,w-1})}{W-1}
\end{equation}

\begin{equation}
\text{ScoreVolatility}_i = \sqrt{\frac{1}{W}\sum_{w=1}^{W} (\text{Score}_{i,w} - \bar{S}_i)^2}
\end{equation}

%==============================================================================
\section{Mixed-Effects Modeling}
%==============================================================================

\subsection{Model Specification}

To decompose variance, we estimate:

\begin{equation}
\text{Placement}_{ij} = \beta_0 + \boldsymbol{\beta}^T \mathbf{X}_i + u_j + \epsilon_{ij}
\end{equation}

where:
\begin{itemize}
    \item $i$ indexes celebrities, $j$ indexes professional dancers
    \item $\mathbf{X}_i$ is the celebrity feature vector
    \item $u_j \sim N(0, \sigma_u^2)$ is the random intercept for pro dancer $j$
    \item $\epsilon_{ij} \sim N(0, \sigma_\epsilon^2)$ is the residual error
\end{itemize}

\subsection{Intraclass Correlation Coefficient (ICC)}

The ICC quantifies the proportion of variance attributable to the professional dancer:

\begin{equation}
\text{ICC} = \frac{\sigma_u^2}{\sigma_u^2 + \sigma_\epsilon^2}
\end{equation}

\subsubsection{Results}

\begin{center}
\begin{tabular}{lcc}
\toprule
\textbf{Variance Component} & \textbf{Estimate} & \textbf{Interpretation} \\
\midrule
$\sigma_u^2$ (Pro dancer) & 0.41 & Between-pro variance \\
$\sigma_\epsilon^2$ (Residual) & 7.35 & Within-pro variance \\
\textbf{ICC} & \textbf{5.3\%} & Pro explains 5.3\% of variance \\
\bottomrule
\end{tabular}
\end{center}

\colorbox{warnorange!20}{\parbox{\dimexpr\linewidth-2\fboxsep}{
\textbf{Key Finding:} The professional dancer explains only 5.3\% of placement variance. Celebrity characteristics and residual factors dominate. This suggests \textit{who} the celebrity is matters more than \textit{who} they're paired with.
}}

\subsection{Fixed Effects: Celebrity Characteristics}

The celebrity-only model (excluding pro random effects) yielded:

\begin{center}
\begin{tabular}{lccc}
\toprule
\textbf{Predictor} & \textbf{Coefficient} & \textbf{Std. Error} & \textbf{p-value} \\
\midrule
Intercept & 12.45 & 1.23 & $<$0.001 \\
Age & 0.08 & 0.02 & $<$0.001 \\
Star Power Index & -0.15 & 0.04 & $<$0.001 \\
First Week Score & -0.12 & 0.02 & $<$0.001 \\
Industry: Sports & -1.82 & 0.58 & $<$0.01 \\
\bottomrule
\end{tabular}
\end{center}

\textbf{Interpretation:}
\begin{itemize}
    \item \textbf{Age:} Each additional year increases expected placement by 0.08 (older = worse)
    \item \textbf{Star Power:} Higher fame predicts better placement (fan voting advantage)
    \item \textbf{First Week Score:} Strong start predicts better final placement
    \item \textbf{Sports Industry:} Athletes perform better than entertainment industry average
\end{itemize}

%==============================================================================
\section{Propensity Score Matching}
%==============================================================================

\subsection{Motivation}

Mixed-effects models estimate correlations but cannot establish causation due to non-random pro assignment. We use \textbf{propensity score matching} to create comparable groups.

\subsection{Methodology}

\subsubsection{Step 1: Define Treatment}
We classify pros as ``top-tier'' based on historical performance:
\begin{equation}
\text{TopPro}_j = \mathbf{1}\{\text{WinRate}_j > \text{Median}(\text{WinRate})\}
\end{equation}

\subsubsection{Step 2: Estimate Propensity Scores}
Using logistic regression:
\begin{equation}
P(\text{TopPro} = 1 \mid \mathbf{X}) = \frac{1}{1 + e^{-(\alpha + \boldsymbol{\gamma}^T \mathbf{X})}}
\end{equation}

Features included: celebrity age, star power, industry, region, season.

\subsubsection{Step 3: Match on Propensity Scores}
Nearest-neighbor matching with caliper = 0.15:
\begin{itemize}
    \item Original sample: 421 celebrities
    \item Matched pairs: 156
    \item Balance achieved: standardized mean differences $<$ 0.1 for all covariates
\end{itemize}

\subsection{Average Treatment Effect (ATE)}

\begin{equation}
\text{ATE} = \frac{1}{n_{\text{matched}}}\sum_{i \in \text{matched}} \left(\text{Placement}_i^{\text{treated}} - \text{Placement}_i^{\text{control}}\right)
\end{equation}

\begin{center}
\colorbox{successgreen!20}{\parbox{0.8\linewidth}{
\centering
\textbf{ATE = --1.70 places} \\
(95\% CI: --2.45 to --0.95, $p < 0.001$) \\[0.5em]
Being paired with a top-tier professional improves final placement by 1.70 places on average, after controlling for celebrity characteristics.
}}
\end{center}

\subsection{Robustness Checks}

We verified results with alternative specifications:

\begin{center}
\begin{tabular}{lcc}
\toprule
\textbf{Specification} & \textbf{ATE} & \textbf{95\% CI} \\
\midrule
Main specification & --1.70 & [--2.45, --0.95] \\
Caliper = 0.10 & --1.58 & [--2.38, --0.78] \\
Caliper = 0.20 & --1.75 & [--2.41, --1.09] \\
Mahalanobis matching & --1.82 & [--2.64, --1.00] \\
\bottomrule
\end{tabular}
\end{center}

%==============================================================================
\section{Pro Dancer Impact Score (PDIS)}
%==============================================================================

\subsection{Conceptual Framework}

The PDIS is a counterfactual metric: \textit{How does each pro's actual performance compare to what would be expected given their celebrities?}

\subsection{PDIS Calculation}

\subsubsection{Step 1: Estimate Expected Placement}

For each celebrity $i$, we predict expected placement using only celebrity features:

\begin{equation}
\hat{P}_i = f(\mathbf{X}_i; \hat{\boldsymbol{\theta}})
\end{equation}

where $f$ is a Random Forest regressor trained on celebrity features only.

\subsubsection{Step 2: Compute Raw PDIS}

For each pro $j$ with partnerships $\mathcal{C}_j$:

\begin{equation}
\text{PDIS}_j^{\text{raw}} = \frac{1}{|\mathcal{C}_j|}\sum_{i \in \mathcal{C}_j} (\hat{P}_i - P_i)
\end{equation}

Positive values indicate the pro outperformed expectations.

\subsubsection{Step 3: Normalize to 0--100 Scale}

\begin{equation}
\text{PDIS}_j = 100 \times \frac{\text{PDIS}_j^{\text{raw}} - \text{PDIS}_{\min}}{\text{PDIS}_{\max} - \text{PDIS}_{\min}}
\end{equation}

\subsection{PDIS Rankings}

\begin{center}
\begin{tabular}{clcccc}
\toprule
\textbf{Rank} & \textbf{Professional} & \textbf{PDIS} & \textbf{Seasons} & \textbf{Wins} & \textbf{Avg Place} \\
\midrule
1 & Daniella Karagach & 100.0 & 5 & 1 & 3.2 \\
2 & Lindsay Arnold & 88.0 & 9 & 1 & 4.1 \\
3 & Derek Hough & 82.3 & 17 & 6 & 3.8 \\
4 & Witney Carson & 78.5 & 10 & 2 & 4.5 \\
5 & Julianne Hough & 75.2 & 4 & 2 & 2.8 \\
6 & Kym Johnson & 72.8 & 13 & 2 & 4.2 \\
7 & Peta Murgatroyd & 70.1 & 11 & 2 & 4.9 \\
8 & Mark Ballas & 68.4 & 19 & 2 & 4.6 \\
9 & Cheryl Burke & 65.7 & 26 & 2 & 5.1 \\
10 & Maksim Chmerkovskiy & 62.3 & 18 & 1 & 5.4 \\
\bottomrule
\end{tabular}
\end{center}

\subsection{PDIS Validation}

We validated PDIS against external benchmarks:

\begin{itemize}
    \item \textbf{Correlation with win count:} $r = 0.72$ ($p < 0.001$)
    \item \textbf{Correlation with experience:} $r = 0.18$ (weak -- PDIS captures skill beyond experience)
    \item \textbf{Face validity:} Top PDIS pros are widely recognized as elite (Derek Hough, Julianne Hough)
\end{itemize}

\subsection{Tier Classification}

Based on PDIS, we classified pros into performance tiers:

\begin{center}
\begin{tabular}{llcc}
\toprule
\textbf{Tier} & \textbf{PDIS Range} & \textbf{Count} & \textbf{Avg Placement} \\
\midrule
Elite & $\geq 75$ & 5 & 3.9 \\
Above Average & 50--75 & 8 & 4.8 \\
Average & 25--50 & 11 & 6.2 \\
Below Average & $<25$ & 8 & 7.8 \\
\bottomrule
\end{tabular}
\end{center}

%==============================================================================
\section{Celebrity Archetype Clustering}
%==============================================================================

\subsection{Motivation}

Different celebrities have different success patterns. By clustering celebrities into archetypes, we can identify which professional dancer style complements each archetype.

\subsection{Methodology}

\subsubsection{Feature Selection}
We selected 7 features for clustering:
\begin{enumerate}
    \item Age at competition
    \item Star Power Index
    \item First week score
    \item Improvement rate
    \item Score volatility
    \item Industry (encoded)
    \item Region (encoded)
\end{enumerate}

\subsubsection{Clustering Algorithm}
K-Means clustering with:
\begin{itemize}
    \item Standardized features (z-score normalization)
    \item Optimal $k$ determined by elbow method + silhouette score
    \item Optimal $k = 3$--$4$ (silhouette = 0.174--0.182)
\end{itemize}

\subsection{Archetype Profiles}

We identified three distinct celebrity archetypes:

\begin{center}
\begin{tabular}{lccccl}
\toprule
\textbf{Archetype} & \textbf{Count} & \textbf{Avg Age} & \textbf{Star Power} & \textbf{Avg Place} & \textbf{Profile} \\
\midrule
Fast Learners & 178 & 34 & 6.3 & 5.9 & Young, athletic, rapid improvement \\
Underdogs & 143 & 34 & 7.9 & 5.7 & Entertainment industry, fan favorites \\
Veteran Stars & 76 & 56 & 15.1 & 9.7 & Older, famous, lower placement \\
\bottomrule
\end{tabular}
\end{center}

\subsection{Archetype Descriptions}

\subsubsection{Fast Learners (45\%)}
\begin{itemize}
    \item \textbf{Demographics:} Younger (avg age 34), athletic background common
    \item \textbf{Industry:} 50\% Sports, 43\% Entertainment
    \item \textbf{Pattern:} Strong improvement rate, moderate starting scores
    \item \textbf{Best Pros:} Kym Johnson (avg placement 2.8), Derek Hough (avg 2.9)
\end{itemize}

\subsubsection{Underdogs (36\%)}
\begin{itemize}
    \item \textbf{Demographics:} Diverse ages, moderate fame
    \item \textbf{Industry:} 89\% Entertainment (actors, singers)
    \item \textbf{Pattern:} Build fan following over season, surprise eliminations
    \item \textbf{Best Pros:} Lindsay Arnold (avg placement 2.5), Derek Hough (avg 3.2)
\end{itemize}

\subsubsection{Veteran Stars (19\%)}
\begin{itemize}
    \item \textbf{Demographics:} Older (avg age 56), high star power
    \item \textbf{Industry:} Diverse -- legacy celebrities, politicians
    \item \textbf{Pattern:} High initial interest, fade as season progresses
    \item \textbf{Best Pros:} Edyta Sliwinska (avg placement 7.0), patient/supportive style
\end{itemize}

\subsection{Archetype-Pro Pairing Recommendations}

Based on historical performance patterns:

\begin{center}
\begin{tabular}{llp{6cm}}
\toprule
\textbf{Archetype} & \textbf{Recommended Pros} & \textbf{Rationale} \\
\midrule
Fast Learners & Derek Hough, Kym Johnson, Witney Carson & Can push technically, leverage athleticism \\
Underdogs & Lindsay Arnold, Peta Murgatroyd & Build confidence, fan connection focus \\
Veteran Stars & Edyta Sliwinska, Emma Slater & Patient teaching style, adapt to limitations \\
\bottomrule
\end{tabular}
\end{center}

%==============================================================================
\section{Dual-Pathway Analysis}
%==============================================================================

\subsection{Judges vs. Fans}

We explored whether pro dancer effects differ for judge scores vs. fan votes:

\begin{center}
\begin{tabular}{lcc}
\toprule
\textbf{Outcome} & \textbf{Pro ICC} & \textbf{Celebrity $R^2$} \\
\midrule
Final Placement & 5.3\% & 26.2\% \\
Judge Scores (avg) & 7.8\% & 31.5\% \\
Fan Vote Estimate & 3.1\% & 22.8\% \\
\bottomrule
\end{tabular}
\end{center}

\textbf{Key Insight:} Pro dancers have more influence on judge scores (7.8\%) than fan votes (3.1\%). This is intuitive: judges evaluate technical execution (influenced by pro's teaching), while fans vote based on likability (driven by celebrity characteristics).

%==============================================================================
\section{Conclusions \& Recommendations}
%==============================================================================

\subsection{Summary of Findings}

\begin{enumerate}
    \item \textbf{Pro dancer impact is modest:} ICC = 5.3\%, but top pros provide ~1.7 place advantage
    \item \textbf{Celebrity characteristics dominate:} Age, fame, and baseline skill explain 26\% of placement variance
    \item \textbf{PDIS identifies elite pros:} Daniella Karagach, Lindsay Arnold, Derek Hough consistently outperform expectations
    \item \textbf{Archetypes enable strategic pairing:} Fast Learners, Underdogs, and Veteran Stars have different optimal pro matches
\end{enumerate}

\subsection{Practical Recommendations for Producers}

\begin{enumerate}
    \item \textbf{Strategic Pairing:} Match Fast Learners with technically demanding pros; pair Veteran Stars with patient teachers
    \item \textbf{Risk Management:} Underdogs are high-variance -- pair with consistent performers to reduce early elimination risk
    \item \textbf{PDIS for Decision-Making:} Use PDIS rather than win count to evaluate pro performance
\end{enumerate}

\subsection{Limitations}

\begin{enumerate}
    \item \textbf{Sample size:} Some pros have few partnerships, making PDIS estimates noisy
    \item \textbf{Endogeneity:} Despite propensity matching, unobserved confounders may bias estimates
    \item \textbf{Temporal changes:} Voting rules and show format changed over 34 seasons
    \item \textbf{Feature limitations:} Star Power Index is a proxy; actual social media data would improve accuracy
\end{enumerate}

\subsection{Future Work}

\begin{itemize}
    \item Incorporate dance style-specific effects (Latin vs. Ballroom)
    \item Model elimination timing (early vs. late season dynamics)
    \item Develop pro-celebrity compatibility score using collaborative filtering
\end{itemize}

%==============================================================================
\section{Appendix: Mathematical Notation}
%==============================================================================

\begin{center}
\begin{tabular}{ll}
\toprule
\textbf{Symbol} & \textbf{Definition} \\
\midrule
$i$ & Celebrity index \\
$j$ & Professional dancer index \\
$P_i$ & Final placement of celebrity $i$ \\
$\hat{P}_i$ & Expected placement based on celebrity features only \\
$\mathbf{X}_i$ & Celebrity feature vector \\
$u_j$ & Random intercept for pro dancer $j$ \\
$\sigma_u^2$ & Between-pro variance \\
$\sigma_\epsilon^2$ & Residual (within-pro) variance \\
ICC & Intraclass Correlation Coefficient \\
ATE & Average Treatment Effect \\
PDIS & Pro Dancer Impact Score \\
SPI & Star Power Index \\
\bottomrule
\end{tabular}
\end{center}

%==============================================================================
\section{Appendix: Output Files}
%==============================================================================

All analysis outputs are saved in \texttt{data/obj3/}:

\begin{itemize}
    \item \texttt{celebrity\_features.csv} -- Complete feature dataset (421 celebrities × 21 features)
    \item \texttt{pro\_dancer\_stats.csv} -- Pro dancer statistics (60 pros)
    \item \texttt{pdis\_final\_rankings.csv} -- PDIS scores for all pros
    \item \texttt{celebrity\_archetypes.csv} -- Cluster assignments for all celebrities
    \item \texttt{cluster\_optimization.png} -- Elbow and silhouette plots
    \item \texttt{celebrity\_archetypes.png} -- PCA visualization of clusters
    \item \texttt{archetype\_pro\_heatmap.png} -- Pro-archetype performance matrix
    \item \texttt{archetype\_industry\_breakdown.png} -- Industry distribution by archetype
\end{itemize}

\end{document}

\documentclass[12pt,letterpaper]{article}
\usepackage[margin=1in]{geometry}
\usepackage{amsmath,amssymb,amsthm}
\usepackage{graphicx}
\usepackage{booktabs}
\usepackage{array}
\usepackage{hyperref}
\usepackage{xcolor}
\usepackage{algorithm}
\usepackage{algpseudocode}
\usepackage{enumitem}
\usepackage{float}
\usepackage{fancyhdr}
\usepackage{titlesec}

% Custom colors
\definecolor{dwtsblue}{RGB}{70,130,180}
\definecolor{dwtscoral}{RGB}{255,127,80}

% Page style
\pagestyle{fancy}
\fancyhf{}
\rhead{MCM Problem C 2026}
\lhead{Objective 1: Fan Vote Estimation}
\rfoot{Page \thepage}

% Title formatting
\titleformat{\section}{\Large\bfseries\color{dwtsblue}}{\thesection}{1em}{}
\titleformat{\subsection}{\large\bfseries\color{gray}}{\thesubsection}{1em}{}

\title{\textbf{Dancing with the Stars: Fan Vote Estimation Model} \\ 
\large MCM Problem C 2026 -- Objective 1 Technical Documentation}
\author{Team Documentation}
\date{January 2026}

\begin{document}

\maketitle

\tableofcontents
\newpage

%==============================================================================
\section{Executive Summary}
%==============================================================================

This document details our mathematical framework for estimating unknown fan votes in \textit{Dancing with the Stars} (DWTS). We developed two complementary optimization models:

\begin{enumerate}[label=\arabic*.]
    \item \textbf{Rank-Based Method} (Seasons 1--2, 28--34): Monte Carlo sampling with feasibility constraints
    \item \textbf{Percent-Based Method} (Seasons 3--27): Convex optimization using CVXPY
\end{enumerate}

\subsection*{Key Results}
\begin{itemize}
    \item Generated \textbf{2,067 fan vote estimates} for 391 contestants across 34 seasons
    \item Achieved \textbf{100\% elimination consistency} (249/249 weeks correctly predicted)
    \item Rank method shows \textbf{98.2\% stability} vs. percent method's 41.4\% under 10\% noise
    \item Mean uncertainty ratio: \textbf{3.676} (range of feasible solutions / point estimate)
\end{itemize}

%==============================================================================
\section{Problem Formulation}
%==============================================================================

\subsection{The Inverse Problem}

Fan votes in DWTS are a ``closely guarded secret.'' We observe:
\begin{itemize}
    \item Judge scores $J_i^{(w)}$ for contestant $i$ in week $w$
    \item Elimination outcomes $E^{(w)}$ (who was eliminated each week)
    \item Final placements
\end{itemize}

Our goal is to \textbf{invert the voting mechanism} to estimate fan votes $F_i^{(w)}$ that are consistent with observed eliminations.

\subsection{Notation}

Let $N^{(w)}$ denote the number of contestants remaining in week $w$. For each week $w$:

\begin{table}[H]
\centering
\begin{tabular}{cl}
\toprule
\textbf{Symbol} & \textbf{Description} \\
\midrule
$J_i^{(w)}$ & Total judge score for contestant $i$ in week $w$ \\
$F_i^{(w)}$ & Fan votes for contestant $i$ in week $w$ (unknown) \\
$R_i^{J}$ & Judge score rank for contestant $i$ (1 = highest score) \\
$R_i^{F}$ & Fan vote rank for contestant $i$ (1 = most votes) \\
$P_i^{J}$ & Judge score percentage for contestant $i$ \\
$P_i^{F}$ & Fan vote percentage for contestant $i$ \\
$E^{(w)}$ & Index of eliminated contestant in week $w$ \\
\bottomrule
\end{tabular}
\caption{Notation used throughout this document}
\end{table}

%==============================================================================
\section{Mathematical Framework}
%==============================================================================

\subsection{Method 1: Rank-Based Combination}

\textbf{Used in:} Seasons 1--2 and 28--34

In this method, contestants are ranked by judge scores and fan votes separately, then ranks are summed. The contestant with the \textbf{highest combined rank} (worst performance) is eliminated.

\subsubsection{Ranking Function}

For judge scores, we compute ranks in descending order (highest score = rank 1):
\begin{equation}
R_i^{J} = \text{rank}\left(-J_i^{(w)}\right), \quad i = 1, \ldots, N^{(w)}
\end{equation}

For ties, we use the \textbf{average rank method}:
\begin{equation}
\text{If } J_a = J_b \text{ and both would occupy ranks } k \text{ and } k+1, \text{ then } R_a^{J} = R_b^{J} = \frac{k + (k+1)}{2} = k + 0.5
\end{equation}

Similarly for fan votes (higher votes = rank 1):
\begin{equation}
R_i^{F} = \text{rank}\left(-F_i^{(w)}\right)
\end{equation}

\subsubsection{Combined Score}

The combined rank score is:
\begin{equation}
S_i = R_i^{J} + R_i^{F}
\end{equation}

\subsubsection{Elimination Constraint}

The eliminated contestant $e = E^{(w)}$ must satisfy:
\begin{equation}
\boxed{S_e = \max_{i \in \{1, \ldots, N^{(w)}\}} S_i}
\end{equation}

That is, the eliminated contestant has the \textbf{worst (highest) combined rank sum}.

\subsubsection{Optimization Problem}

Since ranking is a non-convex operation, we use \textbf{Monte Carlo sampling}:

\begin{algorithm}[H]
\caption{Rank-Based Fan Vote Estimation}
\begin{algorithmic}[1]
\Require Judge scores $\{J_1, \ldots, J_N\}$, eliminated index $e$
\Ensure Fan vote distribution $\{F_1, \ldots, F_N\}$
\State $\text{valid\_solutions} \gets \emptyset$
\For{$k = 1$ to $K$ iterations}
    \State Sample $\vec{\alpha} \sim \text{Dirichlet}(\mathbf{1}_N)$ \Comment{Random proportions}
    \State $F_i \gets \alpha_i \times 10^7$ for all $i$ \Comment{Scale to realistic vote counts}
    \State Compute $R_i^{J}, R_i^{F}, S_i$ for all $i$
    \If{$\arg\max_i S_i = e$}
        \State $\text{margin} \gets S_e - \max_{j \neq e} S_j$
        \State $\text{valid\_solutions} \gets \text{valid\_solutions} \cup \{(\vec{F}, \text{margin})\}$
    \EndIf
\EndFor
\State \Return solution with maximum margin from valid\_solutions
\end{algorithmic}
\end{algorithm}

\textbf{Rationale for Dirichlet prior:} The Dirichlet distribution with parameter $\alpha = \mathbf{1}$ is the uniform distribution over the $(N-1)$-simplex, ensuring unbiased sampling of all possible fan vote proportions.

%------------------------------------------------------------------------------
\subsection{Method 2: Percent-Based Combination}
%------------------------------------------------------------------------------

\textbf{Used in:} Seasons 3--27

In this method, judge scores and fan votes are converted to percentages of the weekly total, then summed. The contestant with the \textbf{lowest combined percentage} is eliminated.

\subsubsection{Percentage Calculations}

Judge score percentage:
\begin{equation}
P_i^{J} = \frac{J_i^{(w)}}{\displaystyle\sum_{j=1}^{N^{(w)}} J_j^{(w)}}
\end{equation}

Fan vote percentage:
\begin{equation}
P_i^{F} = \frac{F_i^{(w)}}{\displaystyle\sum_{j=1}^{N^{(w)}} F_j^{(w)}}
\end{equation}

\subsubsection{Combined Score}

The combined percentage score is:
\begin{equation}
S_i = P_i^{J} + P_i^{F}
\end{equation}

Note that $\sum_i P_i^{J} = 1$ and $\sum_i P_i^{F} = 1$, so $\sum_i S_i = 2$.

\subsubsection{Elimination Constraint}

The eliminated contestant $e = E^{(w)}$ must satisfy:
\begin{equation}
\boxed{S_e = \min_{i \in \{1, \ldots, N^{(w)}\}} S_i}
\end{equation}

\subsubsection{Convex Optimization Formulation}

Since percentages are linear in fan votes, we can formulate this as a \textbf{convex optimization problem}:

\begin{equation}
\begin{aligned}
\underset{\vec{F} \in \mathbb{R}^N}{\text{minimize}} \quad & \left\| \vec{F} - \vec{F}^{\text{prior}} \right\|_2^2 \\
\text{subject to} \quad & \sum_{i=1}^{N} F_i = 1 \quad \text{(normalization)} \\
& F_i \geq 0 \quad \forall i \quad \text{(non-negativity)} \\
& P_e^{J} + F_e \leq P_j^{J} + F_j - \epsilon \quad \forall j \neq e \quad \text{(elimination)}
\end{aligned}
\end{equation}

where:
\begin{itemize}
    \item $\vec{F}^{\text{prior}} = \frac{1}{N}\mathbf{1}$ is the uniform prior (maximum entropy)
    \item $\epsilon = 0.001$ is a small margin ensuring strict inequality
    \item We work with normalized proportions ($\sum F_i = 1$), then scale to vote counts
\end{itemize}

\subsubsection{Key Insight: Convexity}

The objective $\|\vec{F} - \vec{F}^{\text{prior}}\|_2^2$ is convex (quadratic). All constraints are linear:
\begin{itemize}
    \item The normalization constraint $\sum F_i = 1$ is an affine equality
    \item Non-negativity constraints $F_i \geq 0$ are linear inequalities
    \item Elimination constraints $F_e - F_j \leq P_j^{J} - P_e^{J} - \epsilon$ are linear
\end{itemize}

Thus, the feasible region is a \textbf{convex polytope}, and we can solve efficiently using interior-point methods (ECOS solver).

%==============================================================================
\section{Uncertainty Quantification}
%==============================================================================

\subsection{The Non-Uniqueness Problem}

A critical observation: \textbf{there are infinitely many fan vote distributions} that produce the same elimination outcome. Our goal is to quantify this uncertainty.

\subsection{Feasible Region Characterization}

For each week, the set of feasible fan vote proportions forms a convex polytope:
\begin{equation}
\mathcal{F} = \left\{ \vec{F} \in \Delta^{N-1} : S_e(\vec{F}) \leq S_j(\vec{F}) - \epsilon, \; \forall j \neq e \right\}
\end{equation}

where $\Delta^{N-1}$ is the $(N-1)$-dimensional probability simplex.

\subsection{Monte Carlo Uncertainty Bounds}

We estimate bounds by sampling uniformly from the simplex:

\begin{algorithm}[H]
\caption{Uncertainty Bound Estimation}
\begin{algorithmic}[1]
\Require Judge percentages $\{P_1^{J}, \ldots, P_N^{J}\}$, eliminated index $e$
\Ensure Min/max feasible fan votes for each contestant
\State $\text{feasible} \gets \emptyset$
\For{$k = 1$ to $M$ samples}
    \State Sample $\vec{F} \sim \text{Dirichlet}(\mathbf{1}_N)$
    \State Compute $S_i = P_i^{J} + F_i$ for all $i$
    \If{$\arg\min_i S_i = e$} \Comment{Correct elimination}
        \State $\text{feasible} \gets \text{feasible} \cup \{\vec{F}\}$
    \EndIf
\EndFor
\State $F_i^{\min} \gets \min_{\vec{F} \in \text{feasible}} F_i$ for all $i$
\State $F_i^{\max} \gets \max_{\vec{F} \in \text{feasible}} F_i$ for all $i$
\State \Return bounds and sample statistics
\end{algorithmic}
\end{algorithm}

\subsection{Uncertainty Metrics}

We define several metrics to quantify uncertainty:

\begin{enumerate}
    \item \textbf{Uncertainty Ratio:}
    \begin{equation}
    U_i = \frac{F_i^{\max} - F_i^{\min}}{\hat{F}_i}
    \end{equation}
    where $\hat{F}_i$ is our point estimate. Higher values indicate more uncertainty.
    
    \item \textbf{Relative Standard Deviation:}
    \begin{equation}
    \sigma_{\text{rel},i} = \frac{\text{std}(F_i | \vec{F} \in \mathcal{F})}{\hat{F}_i}
    \end{equation}
    
    \item \textbf{Feasibility Rate:}
    \begin{equation}
    r = \frac{|\text{feasible samples}|}{M}
    \end{equation}
    Lower rates indicate tighter constraints (less uncertainty).
\end{enumerate}

\subsection{Key Finding: Uncertainty Varies by Position}

Our analysis reveals that \textbf{uncertainty is not uniform}:
\begin{itemize}
    \item \textbf{Eliminated contestants:} Lower uncertainty (tight constraints force low fan votes)
    \item \textbf{Top performers:} Higher uncertainty (many valid solutions allow varying high votes)
    \item \textbf{Early weeks:} Higher uncertainty (more contestants = larger feasible region)
\end{itemize}

%==============================================================================
\section{Validation Results}
%==============================================================================

\subsection{Elimination Consistency}

We verified that our estimated fan votes produce correct eliminations:

\begin{table}[H]
\centering
\begin{tabular}{lccc}
\toprule
\textbf{Method} & \textbf{Correct} & \textbf{Total Weeks} & \textbf{Accuracy} \\
\midrule
Rank-Based (S1--2, S28--34) & 56 & 56 & 100.0\% \\
Percent-Based (S3--27) & 193 & 193 & 100.0\% \\
\midrule
\textbf{Overall} & \textbf{249} & \textbf{249} & \textbf{100.0\%} \\
\bottomrule
\end{tabular}
\caption{Elimination consistency validation results}
\end{table}

\subsection{Sensitivity Analysis}

We tested model robustness by adding noise to judge scores:

\begin{table}[H]
\centering
\begin{tabular}{ccc}
\toprule
\textbf{Noise Level} & \textbf{Rank Method Stability} & \textbf{Percent Method Stability} \\
\midrule
5\% & 99.1\% & 67.3\% \\
10\% & 98.2\% & 41.4\% \\
15\% & 96.8\% & 28.7\% \\
\bottomrule
\end{tabular}
\caption{Stability under judge score perturbation}
\end{table}

\textbf{Key Insight:} The rank-based method is more robust to score variations because small score changes rarely alter rankings, whereas percentage changes propagate linearly.

%==============================================================================
\section{Controversial Cases Analysis}
%==============================================================================

\subsection{Overview}

We analyzed four historically controversial contestants where fan votes seemingly overruled judge scores:

\begin{table}[H]
\centering
\begin{tabular}{lcccc}
\toprule
\textbf{Contestant} & \textbf{Season} & \textbf{Placement} & \textbf{Est. Fan Vote \%} & \textbf{Advantage} \\
\midrule
Jerry Rice & 2 & 2nd & 20.3\% & +4.9\% \\
Billy Ray Cyrus & 4 & 5th & 13.6\% & +0.2\% \\
Bristol Palin & 11 & 3rd & 15.2\% & +1.8\% \\
Bobby Bones & 27 & 1st & 11.6\% & +0.9\% \\
\bottomrule
\end{tabular}
\caption{Controversial contestants and their estimated fan vote advantages}
\end{table}

\subsection{Bristol Palin: Most Extreme Case}

Bristol Palin (Season 11) represents the most extreme fan-judge disconnect:
\begin{itemize}
    \item Had the \textbf{lowest judge scores 12 times} throughout the season
    \item Still finished \textbf{3rd place}
    \item Our model estimates she had the \textbf{\#1 average fan votes} in Season 11 (15.2\%)
\end{itemize}

This validates our model: the only way to explain her survival is through extraordinarily high fan support.

\subsection{Bobby Bones: Voting Method Impact}

Bobby Bones (Season 27) won despite consistently low judge scores:
\begin{itemize}
    \item Season 27 used the \textbf{rank-based method}
    \item In rank-based voting, even a 1st-place fan rank provides significant advantage
    \item As a radio host reaching 150+ stations, he had a built-in voting army
\end{itemize}

This highlights how the \textbf{choice of voting method significantly impacts outcomes}.

%==============================================================================
\section{Implementation Details}
%==============================================================================

\subsection{Software Stack}

\begin{itemize}
    \item \textbf{CVXPY 1.8.0:} Convex optimization modeling
    \item \textbf{ECOS/SCS:} Interior-point solvers
    \item \textbf{SciPy 1.17.0:} \texttt{rankdata} with \texttt{method='average'}
    \item \textbf{NumPy/Pandas:} Data manipulation
    \item \textbf{Matplotlib/Seaborn:} Visualization
\end{itemize}

\subsection{Computational Performance}

\begin{itemize}
    \item \textbf{Total runtime:} $\sim$2 minutes for all 34 seasons
    \item \textbf{Success rate:} 99.2\% (2,067/2,084 contestant-weeks)
    \item \textbf{Optimization convergence:} Typically 10--50 iterations per week
\end{itemize}

%==============================================================================
\section{Future Work: Ideas for Objectives 2--5}
%==============================================================================

\subsection{Objective 2: Rank vs. Percent Method Comparison}

\textbf{Questions to Address:}
\begin{enumerate}
    \item Apply \textit{both} methods to \textit{all} seasons --- how often do outcomes differ?
    \item Does one method favor fan votes more than the other?
    \item Simulate the ``judges choose from bottom two'' rule (S28+ modification)
\end{enumerate}

\textbf{Proposed Approach:}
\begin{itemize}
    \item For each season/week, compute eliminations under both methods
    \item Create a ``counterfactual history'' --- who would have been eliminated differently?
    \item Quantify fan vote leverage: define a metric $\lambda = \frac{\partial \text{elimination}}{\partial \text{fan vote}}$
\end{itemize}

\subsection{Objective 3: Pro Dancer and Celebrity Characteristics Analysis}

\textbf{Questions to Address:}
\begin{enumerate}
    \item How much does the professional partner impact performance?
    \item Do celebrity characteristics (age, industry, homestate) predict success?
    \item Are effects different for judge scores vs. fan votes?
\end{enumerate}

\textbf{Proposed Approach:}
\begin{itemize}
    \item \textbf{Mixed-effects regression:}
    \begin{equation}
    \text{Score}_{ij} = \beta_0 + \beta_1 \cdot \text{Age}_i + \beta_2 \cdot \text{Industry}_i + u_j + \epsilon_{ij}
    \end{equation}
    where $u_j$ is a random effect for pro dancer $j$
    
    \item \textbf{Causal inference:} Use propensity score matching to estimate pro dancer effect
    
    \item \textbf{Feature importance:} Random forest or gradient boosting to rank predictors
\end{itemize}

\subsection{Objective 4: Alternative Voting System Design}

\textbf{Goal:} Propose a ``fairer'' or ``more exciting'' voting system

\textbf{Ideas to Explore:}

\begin{enumerate}
    \item \textbf{Weighted combination:}
    \begin{equation}
    S_i = \alpha \cdot P_i^{J} + (1-\alpha) \cdot P_i^{F}
    \end{equation}
    Optimize $\alpha$ to balance skill recognition vs. fan engagement
    
    \item \textbf{Momentum-adjusted scoring:}
    \begin{equation}
    S_i^{(w)} = P_i^{J} + P_i^{F} + \gamma \cdot \Delta_i^{(w)}
    \end{equation}
    where $\Delta_i^{(w)} = \text{Score}_i^{(w)} - \text{Score}_i^{(w-1)}$ rewards improvement
    
    \item \textbf{Elimination protection:} Top performer each week is immune
    
    \item \textbf{Ranked-choice voting:} Fans rank top 3, instant runoff
    
    \item \textbf{Skill-tiered voting:} Fan votes weighted by judge score bracket
\end{enumerate}

\subsection{Objective 5: Producer Memo}

\textbf{Key Recommendations to Include:}
\begin{enumerate}
    \item Rank vs. percent method tradeoffs
    \item When to use ``judges choose from bottom two''
    \item Balancing competition integrity with entertainment value
    \item Managing controversial outcomes
\end{enumerate}

%==============================================================================
\section{Appendix: Data Summary}
%==============================================================================

\begin{table}[H]
\centering
\begin{tabular}{ll}
\toprule
\textbf{Statistic} & \textbf{Value} \\
\midrule
Total contestants & 421 \\
Seasons covered & 1--34 \\
Fan vote estimates generated & 2,067 \\
Unique pro dancers & 42 \\
Weeks with data & 11 (maximum per season) \\
Judge score columns & 44 \\
\bottomrule
\end{tabular}
\caption{Dataset summary statistics}
\end{table}

\subsection{Voting Method by Season}

\begin{equation}
\text{Method}(s) = \begin{cases}
\text{Rank-based} & \text{if } s \in \{1, 2\} \cup \{28, 29, \ldots, 34\} \\
\text{Percent-based} & \text{if } s \in \{3, 4, \ldots, 27\}
\end{cases}
\end{equation}

%==============================================================================
\section{References}
%==============================================================================

\begin{enumerate}
    \item COMAP MCM 2026 Problem C: ``Data with the Stars''
    \item Diamond, S., \& Boyd, S. (2016). CVXPY: A Python-embedded modeling language for convex optimization. \textit{Journal of Machine Learning Research}.
    \item SciPy documentation: \texttt{scipy.stats.rankdata}
\end{enumerate}

\end{document}
